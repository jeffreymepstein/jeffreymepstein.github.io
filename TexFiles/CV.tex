\documentclass[]{article}
%\renewcommand{\familydefault}{\sfdefault}
\usepackage[margin=1in]{geometry}
\usepackage{amsmath}
\usepackage{amssymb}
%\usepackage{bbold}
\usepackage{fancyhdr}
\newcommand{\brak}[1]{\left\langle#1\right\rangle}
\newcommand{\com}[2]{\left[#1,#2\right]}
\newcommand{\acom}[2]{\left\lbrace#1,#2\right\rbrace}
\newcommand{\inner}[2]{\left\langle#1,#2\right\rangle}
\newcommand{\bd}[1]{\textbf{#1}}
\usepackage{color}
\usepackage{verbatim}
\usepackage{mathtools}



\begin{document}
\noindent \large Jeffrey M. Epstein\vspace{5pt}\hfill\small\\
\textbf{\color{red} center and bold name, put "title"}
jeffrey.m.epstein@gmail.com\\
jeffreymepstein.github.io\\
\\

\textbf{\color{red}Put number and San Francisco}\\
\\
Quantum information theorist with experience in characterization and benchmarking of quantum processors. Particularly interested in approaches that provide actionable feedback for hardware improvement and comprehensive error models that allow evaluation of error correction schemes.\\
\\
\textbf{\color{red}center section headings}
\subsection*{Professional Experience}
\textbf{Atom Computing}; Berkeley, CA\\
Senior Quantum Applications Engineer, August 2023-present \textbf{\color{red}right justify dates, bold job title}\\
Quantum Applications Engineer, August 2021-July 2023
\begin{itemize}
\item[-] Developed circuit-level tools for efficient and informative characterization of single and two-qubit gates. Resulting software package used internally by hardware engineers to perform rigorous analyses of gate performance. Information provided by this tool can be used to inform hardware improvement and circuit simulations.
\item[-] Led characterization/benchmarking component of DARPA US2QC program. Developed broad knowledge of state-of-the-art techniques in the analysis and error modeling of near-term quantum processors, which I presented to the testing and evaluation team composed of experts from government labs.
\item[-] Developed and studied novel state preparation algorithm for constrained optimization, leading to a publication and a patent (pending).
\item[-] Built tools based on Q-CTRL for optimization of pulse sequences on atomic platform, facilitating design of rapid gates robust against various sources of noise.
\item[-] Supervised company's first theory intern, leading to her authorship on a scientific publication.
\end{itemize}
\vspace{6pt}
\noindent\textbf{University of California, Berkeley}; Berkeley, CA\\
Graduate Student Instructor: Physics 112 (intro. to statistical and thermal physics), Physics 7b (intro. thermodynamics and electromagnetism for scientists and engineers) \textbf{\color{red}just put dates and courses 9/14-6/15, put responsibilities (sections, grading, office hours)}
\begin{itemize}
\item[-] Taught sections, held regular office hours, and graded problem sets and exams.
\end{itemize}
\vspace{6pt}
\noindent\textbf{IBM Research, TJ Watson Research Center}; Yorktown Heights, NY\\
Quantum Computing Intern, September 2012-July 2013
\begin{itemize}
\item[-] Studied robustness of randomized benchmarking (RB) under varying noise models, leading to a highly-cited publication used in the field as evidence for the validity of RB for benchmarking quantum processors subject to realistic physical noise.
\end{itemize}





\subsection*{Academic Experience}
\textbf{\color{red}right justify dates}\\
\textbf{National Institute of Standards and Technology}; Gaithersburg, MD\\
NRC postdoctoral scholar, February 2021-June 2021\\
\\
\textbf{University of California, Berkeley}; Berkeley, CA\\
\textbf{PhD}, Physics, December 2020\\
Dissertation: \textit{Statistical Mechanics of Transport Processes in Active Matter}\\
\textbf{MA}, Physics, December 2016\\
\\
\textbf{Perimeter Institute for Theoretical Physics}; Waterloo, ON\\
\textbf{MSc}, Perimeter Scholars International (PSI), June 2014\\
\\
\textbf{Harvard College}; Cambridge, MA\\
\textbf{AB}, Chemistry and Physics, May 2012\\
\textit{magna cum laude} with high honors in field\\
secondary field, Mathematics; language citation, Chinese




\begin{comment}
\subsection*{Research}
I have worked in two main fields: quantum information and nonequilibrium transport.  In quantum information, I have established speed limits on information processing tasks derived from an extension of the Lieb-Robinson bound to time-dependent local Hamiltonians. More recently, I have demonstrated a connection between a class of nonlinear quantum amplifiers and the von Neumann model of measurement and worked on continuous-time quantum error correction. In 2014, as part of the Perimeter Scholars International masters program, I worked with Daniel Gottesman on magic state distillation. From August 2012-July 2013 I worked in the Quantum Information Group at IBM T.J. Watson research center, where I numerically studied the performance of randomized benchmarking, a method for assessing fidelities of quantum gates.\\
\\
Within nonequilibrium transport, I have focused on the emerging field of active matter. In particular, I have shown how active forces appear in the continuum equations of motion of a system of Active Brownian Particles, and I have derived Green-Kubo equations for generic two-dimensional non-equilibrium fluids, including an expression for the odd viscosity.
\end{comment}


\subsection*{Scientific Publications}
\begin{enumerate}

\item \textit{Note on simple and consistent gateset characterization including calibration and decoherence errors}. \textbf{JME}. arXiv:2402.17727 (2024)
	
\item \textit{Subspace Correction for Constraints}. K Pawlak, \textbf{JME}, D Crow, S Gandhari, M Li, T Bohdanowicz, J King arXiv:2310.20191 (2024)

\item \textit{Iterative assembly of} $\prescript{171}{}{\textrm{Yb}}$ \textit{atom arrays in cavity-enhanced optical lattices}. M Norcia \textit{et al.} arXiv:2401.16177 (2024)

\item \textit{Mid-circuit qubit measurement and rearrangement in a} $\prescript{171}{}{\textrm{Yb}}$ \textit{atomic array}. M Norcia \textit{et al.} arXiv:2305.19119 (2023)

\item \textit{Thermally driven quantum refrigerator autonomously resets superconducting qubit}.
M Aamir, P Suria, J Guzmán, C Castillo-Moreno, \textbf{JME}, N Yunger Halpern, S Gasparinetti. arXiv:2305.16710 (2023)
	
\item \textit{Odd Diffusivity of Chiral Random Motion}. C Hargus, \textbf{JME}, KK Mandadapu. Phys. Rev. Lett. 127, 178001 (2021).

\item\textit{Quantum noise limits for a class of nonlinear amplifiers}. \textbf{JME}, KB Whaley, J Combes. Phys. Rev. A 103 (5), 052415 (2021).

\item \textit{Time reversal symmetry breaking and odd viscosity in active fluids: Green-Kubo and NEMD results}. C Hargus, K Klymko, \textbf{JME}, KK Mandadapu.  J. Chem. Phys. 152, 201102 (2020).

\item \textit{Time reversal symmetry breaking in two-dimensional non-equilibrium viscous fluids}. \textbf{JME}, KK Mandadapu. Phys. Rev. E 101, 052614 (2020).

\item \textit{Continuous quantum error correction for evolution under time-dependent Hamiltonians}. J Atalaya, S Zhang, MY Niu, A Babakhani, HCH Chan, \textbf{JME}, KB Whaley. 	arXiv:2003.11248 (2020).

\item  \textit{Statistical Mechanics of Transport Processes in Active Fluids II: Equations of Hydrodynamics for Active Brownian Particles}. \textbf{JME}, K Klymko, KK Mandadapu. J. Chem. Phys. 150, 164111 (2019).

\item \textit{Postponing the orthogonality catastrophe: efficient state preparation for electronic structure simulations on quantum devices}. NM Tubman, C Mejuto-Zaera, \textbf{JME}, D Hait, DS Levine, W Huggins, Z Jiang, JR McClean, R Babbush, M Head-Gordon, KB Whaley.	arXiv:1809.05523 (2018).



\item \textit{Quantum Speed Limits for Quantum Information Processing Tasks}. \textbf{JME}, KB Whaley. Phys. Rev. A 95, 042314 (2017).

\item \textit{Investigating the Limits of Randomized Benchmarking Protocols}. \textbf{JME}, AW Cross, E Magesan, and JM Gambetta. 
Phys. Rev. A 89, 062321 (2014) 

\item \textit{CD36 in the periphery and brain
	synergize in stroke injury in hyperlipidemia}. E Kim, M Febbraio, Y Bao, AT Tolhurst, \textbf{JME}, S Cho.  Annals of Neurology. 71(6) (2012)
\end{enumerate}


\subsection*{Awards}
\begin{itemize}
	\item[-] NIST NRC Postdoctoral Research Associateship, 2020-2021
	\item[-] National Defense Science and Engineering Graduate (NDSEG) Fellowship, 2016-2019
\end{itemize}





\begin{comment}

\subsection*{Conferences and Seminars}
\begin{enumerate}	
\item \textit{Nonlinear amplifiers for measurement of normal operators}. Talk presented to IBM Research - Almaden (2021).

\item \textit{Odd Transport in Active Systems}. Informal Statistical Physics Seminar, University of Maryland (2021).
	
\item \textit{Quantum Foundations Seminar}. Seminar series organized and taught at Berkeley (2020).

\item\textit{Active matter, time reversal symmetry breaking, and Onsager reciprocal relations}. Poster presented at Berkeley Statistical Mechanics Meeting (2020).
	
\item\textit{Rheology of 2D Active Fluids}. Berkeley Soft Matter Seminar (2019).	

\item \textit{Quantum noise limits for a class of nonlinear amplifiers}. Talk presented at APS March Meeting (2019).

\item \textit{Transport Processes in Active Fluids}. Talk presented at APS March Meeting (2019).



\item \textit{Continuum Mechanics of Active Brownian Particles}. Poster presented at Gordon Research Conference and Seminar: Complex Active and Adaptive Material Systems (2019).	

\item \textit{Speed Limits for Quantum Control of Local Spin Systems}. Talk presented at APS March Meeting (2018).


\item \textit{Speed Limits for Quantum Control of Local Spin Systems}. Talk presented at SQuInT (2017).


\item Lectures on Statistical Field Theories. School attended, Galileo Galilei Institute for Theoretical Physics, (2017).


\item \textit{Combinatorial Results on the Stabilizer Polytope}. Poster presented at SQuInT (2015).


\end{enumerate}
\end{comment}









\end{document}
